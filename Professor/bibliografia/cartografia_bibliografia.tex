\begingroup
\renewcommand{\chapter}[2]{\normalsize{\HUGE \textbf{Referências Bibliográficas}}}%
\pagebreak
\newgeometry{textwidth=840pt,outer=50pt}

\begin{multicols}{2}

\begin{thebibliography}{10}



\bibitem[Aguirre -2007]{Aguirre}  Aguirre, A. J.; Mello Filho, J.A. de.  
  \newblock {\em Introdução à cartografia.}
  \newblock Universidade Federal de Santa Maria. Santa Maria–RS, 2007.
  

\bibitem[Almeida-2001]{Almeida}  Almeida, Ros{\^a}ngela Doin de.  
  \newblock {\em Do desenho ao mapa: inicia{\c{c}}{\~a}o cartogr{\'a}fica na escola.}
  \newblock S{\~a}o Paulo: Contexto, 2001.
  
  
  
  
  \bibitem[Anderson-2002]{Anderson}  Anderson K.C.; Leinhardt G.
  \newblock {\em  Maps as representations: expert novice comparison of
projection understanding.}
  \newblock Ong Instr 20(3):283–321, 2002.


\bibitem[Avila - 2008]{Avila}  Avila, G.  
  \newblock {\em A Matemática e a Cartografia.}
  \newblock Revista do Professor de Matemática. São Paulo, no 65, p. 4-11.2008.



\bibitem[Battersby-2012]{Battersby}  Battersby, S.E.; Kessler, F.C. 
  \newblock {\em  Cues for interpreting distortion in map projections.}
  \newblock J Geogr 111 (3):93–101, 2012.

\bibitem[BNCC-2018]{BNCC2018}
\newblock{\em Base Nacional Comum Curricular.}
\newblock{ Brasil, 2018.}


\bibitem[Bortolossi-2016]{Bortolossi}  Bortolossi, H.J.; de Almeida R. V.  
  \newblock {\em USANDO O SOFTWARE DYNATLAS PARA EXPLORAR A CARTOGRAFIA COMO UM TEMA INTERDISCIPLINAR NO ENSINO DA MATEMÁTICA E DA GEOGRAFIA.}
  \newblock 2016.
  
\bibitem[Bottomley -2003]{Bottomley}  Bottomley, H.  
  \newblock {\em ABetween the Sinusoidal projection and the Werner: an alternative to the Bonne.}
  \newblock Cybergeo : European Journal of Geography [En ligne], Cartographie, Imagerie, SIG, document 241, mis en ligne le 13 juin 2003.
  
 \bibitem[Braumann-2002]{Braumann}  Braumann, C.
  \newblock {\em Divagações sobre investigação matemática e o seu papel na apren-dizagem da matemática.}
  \newblock As atividades de investigação na aprendizagem da matemática e na for-mação de professores. Lisboa: SEM-SPCE, p. 5 – 24, 2002. 

\bibitem[Brasil-1997]{PCN}   
  \newblock {\em Par{\^a}metros Curriculares Nacionais.}
  \newblock Brasil. 1997.
  
\bibitem[Downs -1991]{Downs}  Downs, R.M.; Liben, L.S.  
  \newblock {\em  The development of expertise in geography: a
cognitive-developmental approach to geographic education.}
  \newblock Ann Assoc Am Geogr 81(2):304–327, 1991


\bibitem[Fenna -2006]{Fenna} Fenna, D.   
  \newblock {\em Cartographic science: a compendium of map projections with derivations.}
  \newblock CRC Press, 2006.

\bibitem[Feeman-2002]{Feeman} Feeman, T.G.   
  \newblock {\em Portraits of the Earth: A mathematician looks at maps.}
  \newblock American Mathematical Soc., 2002.


\bibitem[Grafarend-2014]{Grafarend} Grafarend, E. W.; Krumm, F. W.   
  \newblock {\em  Map projections.}
  \newblock Berlin–Heidelberg: Springer, 2014.
 

\bibitem[Lapaine-2017]{Lapaine} Lapaine, M.; Usery, E. L. (Ed.). 
  \newblock {\em   Choosing a map projection.}
  \newblock Cham: Springer, 2017.
  
  
  
\bibitem[Lunkes-2012]{Lunkes} Lunkes, R. P.; Martins, G. 
  \newblock {\em   Alfabetização cartográfica: um desafio para o ensino de geografia.}
  \newblock Origem não identificada, 2012.


\bibitem[Ludwig-2016]{Ludwig} Ludwig, A. B.; Nascimento, E. 
  \newblock {\em Os conhecimentos cartográficos na prática docente: um estudo com professores de geografia..}
  \newblock Caminhos de geografia, v. 17, n. 60, p. 183-196, 2016.


\bibitem[Menezes-2016]{menezes} De Menezes, P. M. L. ; do Couto Fernandes, M.l. 
  \newblock {\em   Roteiro de cartografia.}
  \newblock Oficina de textos, 2016.



\bibitem[Mulcahy-2001]{Mulcahy} Mulcahy, K. A.; Clarke, K. C.
  \newblock {\em   Symbolization of map projection distortion: a review.}
  \newblock Cartography and geographic information science, v. 28, n. 3, p. 167-182, 2001.

 \bibitem[Nardi-2016]{Nardi}  Nardi ,R.; Carvalho, A.M.P.  
  \newblock {\em Um estudo sobre a evolução das noções de estudantes sobre espaço, forma e força gravitacional do planeta Terra.}
  \newblock Investigações em ensino de ciências, v. 1, n. 2, p. 132-144, 2016.


\bibitem[Nussbaum -1976]{Nussbaum}  Nussbaum , J.; Novak, J.D.  
  \newblock {\em An assessment of childrens concepts of the Earth utilizing structural interwiews}
  \newblock Science Education, v. 60, n. 4, p. 535-550, 1976.

\bibitem[Nussbaum -1999]{Nussbaum1}  Nussbaum , J.  
  \newblock {\em La tierra como cuerpo cósmico.}
  \newblock Ideas científicas en la infancia y la adolescencia,Madrid: Editora Morata, 1999. 
  


\bibitem[Olson -2006]{Olson}  Olson , J.M.  
  \newblock {\em Map projections and the visual detective: how to tell if a map is equal-area, conformal, or neither}
  \newblock J Geogr 105:13–32, 2006


 \bibitem[Ponte-2003]{Ponte}   Ponte, J.P.
  \newblock {\em Investigação sobre investigações matemáticas em Portugal.}
  \newblock Investigar em educação, p. 93-169, 2003.
  

 \bibitem[Ponte-2003]{Ponte1}   Ponte, J.P.; Brocado, J.; Oliveira H.
  \newblock {\em Investigações matemáticas na sala de aula. }
  \newblock Belo Horizonte: Autêntica Editora, 2003.


\bibitem[Snyder-1989]{Snyder}  Snyder, J. P.; Voxland, P. M.  
  \newblock {\em An album of map projections.}
  \newblock P. M.  US Government Printing Office, 1989.
 

\bibitem[Timbó-2001]{Timbó} Timbó, M. A.
  \newblock {\em    Elementos de cartografia.}
  \newblock  Universidade Federal de Minas Gerais. Belo Horizonte, Brasil, 57p, 2001.




\end{thebibliography}
\end{multicols}
\endgroup