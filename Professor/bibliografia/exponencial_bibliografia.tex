\begingroup
\renewcommand{\chapter}[2]{\normalsize{\HUGE \textbf{Referências Bibliográficas}}}%
\pagebreak
\newgeometry{textwidth=840pt,outer=50pt}

\begin{thebibliography}{1}


\bibitem[Alagic-2006]{Alagic2006}   Alagic, M., Palenz, D.
  \newblock {\em Teachers explore linear and exponential growth: Spreadsheets as cognitive tools.}
  \newblock Journal of Technology and Teacher Education. 14(3), 633-649, 2006.
  
  
 \bibitem[Bergsma-2019]{Bergsma2019}   Bergsma, Y. 
  \newblock {\em Students’ Understanding of Exponents and Exponential Functions at a Secondary School in the Northern Netherlands.}
  \newblock Bachelor’s Project Mathematics. University of Groningen. 2019.
  
  \bibitem[Borges-2015]{Borges2015}   Borges, W. A., Oliveira, M.H.P, Dias, M. 
  \newblock {\em Limites de Estudantes do 1º Ano do Ensino Médio na Resolução de Atividade de Potenciação.}
  \newblock Revista do Programa de Pós-Graduação em Educação Matemática da UFMS. Volume 8, Número 16, 2015.


\bibitem[Bush-2015]{Bush2015}   Bush, S., Gibbons, K., Karp, K, Dillon, F. 
  \newblock {\em Epidemics, exponential function and modelling}
  \newblock Mathematics Teaching in the Middle School, Vol. 21, No. 2, pp. 90-97, 2015.
  
  
  \bibitem[Davis-2009]{Davis2009}   Davis, J. 
  \newblock {\em Understanding the influence of two mathematics textbooks on prospective secondary teachers’ knowledge.}
  \newblock Journal of Mathematics Teacher Education, 12, 365 - 389, 2009.
  

\bibitem[Ellis-2012]{Ellis2012}   Ellis, A. B., Ozgur, Z., Kulow, T., Williams, C. C. and Amidon, J.
  \newblock {\em Quantifying exponential growth: The case of the Jactus. In R. Mayes, R. Bonillia, L. L. Hatfield, and S. Belbase (Eds.), Quantitative reasoning: Current state of understanding.}
  \newblock WISDOMe Monographs (Vol. 2, pp. 93–112). Laramie: University of Wyoming, 2012.

\bibitem[Presmeg-2005]{Presmeg2005}   Presmeg, N., Nenduardu, R.
  \newblock {\em An investigation of a pre-service teacher’s use of representations in solving algebraic problems involving exponential relationships.}
  \newblock In H.L. Chick and J.K. Vincent (eds), Proceedings of the 29th PME International Conference, 4, pp. 105-112, 2005.
  
  
   \bibitem[Seidel-2007]{Seidel2007}   Seidel, R. Perencevich, K., Kett, A.
  \newblock {\em From Principles of Learning to Strategies for Instruction with Workbook Companion A Needs-Based Focus on High School Adolescents.}
  \newblock Springer Science+Business Media, LLC, 2007.

  
  \bibitem[Silva-2015]{Silva2015}   Silva, R. J. A.
  \newblock {\em Contexto e aplicações das funções exponenciais no ensino médio: uma abordagem interdisciplinar.}
  \newblock Dissertação (Mestrado em Matemática) - Universidade Estadual do Norte Fluminense Darcy Ribeiro. Centro de Ciência e Tecnologia. Laboratório de Ciências Matemáticas. Campos dos Goytacazes, 2015.
  
  
  \bibitem[Weber-2002]{Weber2002}   Weber, K.
  \newblock {\em  Students’ understanding of exponential and logarithmic functions.}
  \newblock Master International Conference on the Teaching of Mathematics (pp. 1-10). Crete - Greece: University of Crete, 2002.


  \bibitem[Wu2011]{Wu2011}   Wu, H.-H.
  \newblock {\em  The Mis-Education of Mathematics Teachers.}
  \newblock Notices of the American Mathematical Society, 58(3), 372–384, 2011.


\bibitem[BNCC-2018]{BNCC2018}
\newblock{\em Base Nacional Comum Curricular.}
\newblock{ BRASIL, 2018.}




\end{thebibliography}
